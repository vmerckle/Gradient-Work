Discretize mean-field limit using JKO, see if it is similar to GD.

\textbf{Mean-field limit}\citep{chizatGlobalConvergenceGradient2018}: For a sufficiently large width, the training dynamics of a NN can be coupled with the evolution of a probability distribution described by a PDE. \textbf{Pros:} Using a smaller initialization scale than in the NTK regime, feature learning is allowed to happen. \textbf{Cons:} Discretization is not trivial, and PDE are a bit harder than linear models. The theory is lacking a lot of useful results.

\begin{itemize}
	\item \colorbox{Thistle!40}{Block:} obligé de modifier l'algo du papier pour que ça converge, on sait plus trop se qu'il se passe (potentiellement on fait une GD..)
	\item \colorbox{Emerald!30}{Next step:} donner une initialisation spécifique (uniquement une activation) et voir si ça se diffuse en suivant les coûts. Utiliser JKOnet pour tester aussi.
\end{itemize}

\section{Classic setup}

Data $x_j \in \R^d$ and labels $y_j \in \R$, $j=1,..,n$

First layer $w_i \in \R^d$, second layer $\alpha_i \in \R$, $i=1,..,m$

$\step>0$ step-size, $\reg$ regularization

\begin{equation}
	 \mathcal{L}(W, \alpha) = \sum_{j=1}^n \bigg( \underbrace{\sum_{i=1}^m \max(0, w_i^\top x_j) \alpha_i}_{\text{Network's Output}} - y_j \bigg)^2 + \underbrace{\lambda \sum_{i=1}^m \| w_i \|^2_2 + \alpha_i^2}_{\text{Weight Decay}}
\end{equation}


\textbf{Discret time.}

Full-batch gradient descent

\begin{equation}
	(W, \alpha)_{t+1} = (W, \alpha)_t - \step \nabla \mathcal{L}((W, \alpha)_t)
\end{equation}

Implicit

\begin{equation}
	\param_{t+1} = \argmin_{\param} \mathcal{L}(\param) +\frac{1}{2\step} \norm{\param - \param_t}
\end{equation}

\textbf{Continuous time.}

Taking $\step \to 0$, we get the gradient flow: $\frac{\dd \param_t}{\dd t} = - \nabla \mathcal{L}(\param_t)$. We make ReLU differentiable with $\sigma'(0)=0$ as justified in \citep{boursierGradientFlowDynamics2022}.

\section{Using a measure}

\begin{equation}
	\int_\paramS m(\param; x) \dd \mu(\param) = \frac{1}{m} \sum_{i=1}^{m} \inp{\vw_i}{\vx_j}_+ \alpha_i
\end{equation}

Different ways to use a measure:

\begin{itemize}
	\item $\paramS = \R^d \times \R$, measure $\mu = \frac{1}{m} \sum_{i=1}^{m} \dirac_{\param_i = (\vw_i, \alpha_i)}$, output of one neuron $m(\param=(\vw,\alpha); \vx) = \inp{\vx}{\vw}_+ \alpha$: (works, output matches discrete)
	\item $\paramS = \R^d$, measure $\mu = \frac{1}{m} \sum_{i=1}^{m} \alpha_i \dirac_{\param_i = \vw_i}$ output of one neuron $m(\param=\vw; \vx) = \inp{\vx}{\vw}_+$ (works)
	\item $\paramS = \R^d \times \R^d$, output of one neuron $m(\tilde{\vw}_+, \tilde{\vw}_-, \vx) = \inp{\tilde{\vw}_+}{\vx} - \inp{\tilde{\vw}_-}{\vx}$ (works, separate neg and positive)
	\item $\paramS = (S^{d-1} \times \R)$, output of one neuron $m((\vd, \tilde{\alpha}); \vx) = \tilde{\alpha} \inp{\vd}{\vx} = \tilde{\alpha}\  \mathds{1}_{\inp{\vd}{\vx}>0} $ (works), mapping: $\vd = \frac{\vw}{\norm{\vw}}$ and $\tilde{\alpha}=\norm{\vw} \alpha$. Gradient are not equal to discrete.
\end{itemize}

\subsection{Algorithm, discretize the measure's space}

Take a grid of $N$ points in $\paramS$, we can match the notation above by taking a neuron for each point of the grid $m=N$.

\begin{equation}
	\mu(t+1) = \argmin_{\mu \in \mathcal{M}(\paramS)} F(\mu) + \frac{1}{2 \step} W_2(\mu; \mu(t))
\end{equation}

A essayer: KL à la place de distance wasserstein.

Remark: le 1/m c'est principalement pour être ok à l'infini. Dans le papier JKO \citep{carlierConvergenceEntropicSchemes2017} ils utilisent un vecteur de proba

\subsection{Infinity and beyond}

Take $\step \to 0$, get gradient flow. Take $m \to \infty$, get wasserstein gradient flow \citep{bachGradientDescentInfinitely2021}, and if it converges, it goes to the global optimal.

\subsection{JKO}

Use something out of the box to compute JKO flow

\begin{itemize}
	\item \href{https://arxiv.org/pdf/2206.05262.pdf}{Meta Optimal Transport (paper)} and \href{https://github.com/facebookresearch/meta-ot}{(code git)}: InputConvexNN to predict solution of OT problem
	\item \href{https://arxiv.org/pdf/2106.06345.pdf}{JKOnet (paper)} and \href{https://github.com/bunnech/jkonet}{(code git)}:
		\begin{itemize}
			\item \href{https://github.com/bunnech/jkonet/tree/main/jkonet/models}{/models} \verb|->| sinkhorn loss defined in loss.py, differentiable loop in fixed point.py
			\item next step: trying to create the right \href{https://ott-jax.readthedocs.io/en/latest/_autosummary/ott.geometry.geometry.Geometry.html#ott.geometry.geometry.Geometry}{Geometry} object from OTT library, which is what's used for sinkhorn
			\item Or try to use the whole thing?
		\end{itemize}
\end{itemize}

\subsection{Papers}

\href{https://arxiv.org/pdf/1502.06216.pdf}{The algo we try to implement}

Paper with a \href{https://arxiv.org/pdf/1512.02783.pdf}{specific case that doesn't match ours:} 

In the future, \href{https://arxiv.org/pdf/2106.00736.pdf}{large-scale waserstein gradient flows}

\subsubsection{Grid problems}

The grid currently dictate the neuron's scale, giving multiple choices. One solution: duplicate each neuron, make one with a small scale and one with a very big scale.


